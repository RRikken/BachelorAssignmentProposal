\documentclass{article}
\usepackage[T1]{fontenc}
\usepackage[utf8]{inputenc}

\usepackage{lmodern}
\usepackage{verbatim} 
\usepackage{eurosym}
\usepackage[english]{babel}
\usepackage{graphicx}
\usepackage{url}
\usepackage{subcaption}
\usepackage{caption}
\usepackage{enumerate}
\usepackage{float}
\usepackage[top=3cm,bottom=3cm,left=1.9cm,right=1.9cm,columnsep=25pt, asymmetric]{geometry}
\usepackage[titletoc,title]{appendix}
\usepackage{amsmath}
\usepackage{booktabs}
\usepackage{wrapfig}

\renewcommand{\familydefault}{\sfdefault}
\bibliographystyle{plain}

\title{\textbf{Bachelor assignment proposal}}
  
\author{Rob Rikken}
\date{\today}

\begin{document}

\maketitle
\newpage
\tableofcontents
\newpage
\listoffigures
 \newpage
\listoftables
\newpage

\section{Introduction}
The development in satellite technology in the recent decades has led to higher quality data of the earth. 
The satellites now in orbit can identify the car you are driving, what color drink you are drinking, how high your house is. 
This technology is also used in civil engineering to determine ground moisture levels\cite{Karthikeyan2017FourComparisons} and bathymetry\cite{Paloscia2013SoilValidation}. 
Different satellites and algorithms to make sense of the data, have an influence on the accuracy of the measurements.

In the Netherlands, measurements of all types of characteristics of water have a long history. 
This means that satellites do not have the added value they might have in different countries. 
The value of the quality of the measurements in the Netherlands can however, be in the validation of the satellite measurements.
Advances in satellite bathymetry for instance, can be validated with the extensive measurements of all the waterways in the Netherlands.



\section{Research design}
\section{Planning}

\bibliography{references}
\end{document}